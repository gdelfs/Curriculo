\documentclass{tccv}
\usepackage[brazilian]{babel}
\usepackage[utf8]{inputenc}

\title{Gabriel Delfino Resume}
\begin{document}

\part{Gabriel Moysés Delfino}


\section{Experiência}

    \begin{eventlist}

        \event{Agosto 2018 - Outubro2018}
            {BOCOM BBM, Rio de Janeiro, Brasil}
            {DevOps}
            \begin{itemize}
                \item Automatização de processos visando produção acelerada e segura de aplicações e serviços; e
                \item Design, codificação e implementação de API de integração entre JIRA e TFS.
            \end{itemize}
 
        \event{Julho 2017 - Janeiro 2018}
            {Fraunhofer IIS, Ilmenau, Alemanha}
            {Desenvolvedor}
            \begin{itemize}
                \item Desenvolvimento de software em Python para automatização do processo de calibração do equipamento OTA, com o controle remoto simultâneo de 16 geradores de sinal, 6 receptores de sinal e uma unidade de calibração.
            \end{itemize}
        
        \event{Fevereiro 2017 - Janeiro 2018}
            {Escala11, Rio de Janeiro, Brasil}
            {Desenvolvedor}
            \begin{itemize}
                \item Foco no desenvolvimento de novas funcionalidades para o site do \href{https://www.escala11.com/}{Escala11}; e
                \item Atuação no controle de banco de dados, manutenção de serviços, correção de bugs, melhorias de segurança e automatização de processos.
            \end{itemize}
    \end{eventlist}

\section{Pesquisas científicas}

    \begin{itemize}
        \item Pesquisa na área de simulações computacionais em linguagem MCNPX com foco em ações de resposta a acidentes radiológicos (PIBITI 2015/16)  (\href{https://github.com/gdelfs/Simualacoes-Computacionais-em-MCNPX}{\faCode}).
        \item  Pesquisa na área de engenharia nuclear com foco na saúde pública e controle do vetor \textit{Aedes aegypti} através de esterilização por radiação (PIBITI 2016/17) (\href{https://github.com/gdelfs/Esterilizacao-Aedes-Aegypti-com-radiacao}{\faCode}).
    \end{itemize}


\section{Cursos Online}
    \begin{itemize}
        \item HTML, CSS e Javascript para desenvolvimento Web - The Web Developer Bootcamp (Udemy - Fev 2017).  
        \item Reconhecimento Facial em linguagem Python utilizando biblioteca OpenCV (Udemy - Mar 2018).
    \end{itemize}

\personal
    [] % personal website
    {Rio de Janeiro, RJ} % Address
    {+55 (21) 99694 3909} % Phone number
    {gabrielmoysesdelfino@hotmail.com} % Email
    {github.com/gdelfs} % Github


\section{Educação}

\begin{yearlist}

    \item[CR: 7,55 / 10 ]{Dez 2018}
        {Engenharia de Computação}
        {Instituto Militar de Engenharia}

\end{yearlist}

\section{Habilidades técnicas}
    
    \begin{itemize}  
        \item Linguagens de Programação
            \begin{itemize}
                \item Boa experiência: PHP, Python.
                \item Conhecimento intermediário: MCNPX, Java, Android, C/C++, HTML, CSS.
                \item Conhecimento básico: Javascript.
            \end{itemize}
        \item Linguagens de consulta: SQL.
        \item Linguagens de Comando: Bash.
        \item Sistemas Operacionais: Windows, Linux
        \item RDBMS: MySQL 
        \item Controle de versões: GIT, TFS
        \item Acompanhamento de projetos: JIRA
    \end{itemize}

\section{Projetos Acadêmicos}
    
    \begin{itemize}
        \item Desenvolvimento de um sistema completo para apuragem de faltas em sala de aula utilizando reconhecimento facial. Inclui aplicação Android para aquisição da foto, código em python para realizar o reconhecimento e aplicação Web para listagem e edição dos resultados.
        \item Desenvolvimento de um chat entre mobile e computador utilizando comunicação TCP entre as respectivas aplicações Android e Java (\href{https://github.com/gdelfs/Messenger-Android-Computador}{\faCode}).
    \end{itemize}

\section{Idiomas}
    
    \begin{itemize}
        \item Proficiência Avançada em Inglês (TOEFL iBT 2017 111/120; Cambridge FCE 2011).
    \end{itemize}

\section{Informações adicionais}
    
    \begin{itemize}
        \item Curso de Formação de Oficiais do Exército Brasileiro (1º Tenente).
    \end{itemize}
    
\end{document}